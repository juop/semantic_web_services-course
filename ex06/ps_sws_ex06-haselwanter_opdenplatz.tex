\documentclass[a4paper]{article}
\usepackage[top=1in, bottom=1.25in, left=1.25in, right=1.25in]{geometry}
\usepackage[utf8]{inputenc}
\usepackage[english]{babel}
\usepackage[T1]{fontenc}
\usepackage{hyperref}

\title{\textbf{Semantic Web Services}\\\large{Assignment 6}}
\author{Stefan Haselwanter\\Juliette Opdenplatz}


\begin{document}

\maketitle

\section{Question: What is Process Mediation?\\(WSMO, WSML, WSMX stack)}

What is process mediation? Why do we need it? Explain in the context of
semantic web services. Examine the process mediation example in lecture slides
and suggest another scenario.

\section{Answer}

Heterogeneity can and most certainly will be a problem in collaborative tasks 
that depend on a large number of different processes.
There can already be problems if there are just two processes involved, 
that is, one process might depend on data, the other process generates.
This is especially a problem when we think about semantic web service
and how those services might interact with each other.

To overcome interoperability problems like this we need process mediation. 
Heterogeneities can appear in terms of data, ontology, process or protocol,
thus mediators can be distiguished by respective levels.
They might for example mediate either heterogeneous data sources 
or heterogeneous communication patterns.

Of course there are frameworks that offer the ability to implement process mediation. 
The one of most importance is the WSMO approach:
\begin{description}
	\item[Web Service Modelling Ontology (WSMO)] Mediation is a first class citizen.
		WSMO offers specific means to semantically describe concrete mediation 
		solutions and to directly refer to them when needed (e.g. from ontologies or Web services).
	\item[Web Service Modeling Language (WSML)] A language for the WSMO approach.
		\begin{enumerate}
			\item RDFS and OWL have no support for web services, goals or mediators
			\item OWL-S is not expressive enough to cover all aspects of Web Services
		\end{enumerate}
	\item[Web Service Execution Environment (WSMX)] An execution environment to enable the WSMO approach.
		\begin{enumerate}
			\item Service offerings, required capabilities and exchanged data are semantically annotated
			\item An environment to bridge service providers and requesters
			\item Automation of tasks with reasoning
		\end{enumerate}
\end{description}

\end{document}
